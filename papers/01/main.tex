\documentclass[modern]{aastex631}
\usepackage[utf8]{inputenc}
\usepackage{amsmath}

% page and document setup
\renewcommand{\twocolumngrid}{}
\addtolength{\topmargin}{-0.35in}
\addtolength{\textheight}{0.6in}
\setlength{\parindent}{3.5ex}
\renewcommand{\paragraph}[1]{\medskip\par\noindent\textbf{#1}~---}

% figure setup
\usepackage{graphicx}
\usepackage{xcolor}
\usepackage[framemethod=tikz]{mdframed}
\usetikzlibrary{shadows}
\definecolor{captiongray}{HTML}{555555}
\mdfsetup{%
  innertopmargin=2ex,
  innerbottommargin=1.8ex,
  linecolor=captiongray,
  linewidth=0.5pt,
  roundcorner=1pt,
  shadow=false,
}
\newlength{\figurewidth}
\setlength{\figurewidth}{0.75\textwidth}

% text macros
\shorttitle{Astra}
\shortauthors{Casey et al.}
\newcommand{\documentname}{\textsl{Article}}
\newcommand{\sectionname}{Section}
\newcommand{\astra}{\texttt{Astra}}
\newcommand{\Astra}{\astra}
\newcommand{\pipeline}[1]{\texttt{#1}}

\newcommand{\APOGEENet}{\pipeline{APOGEENet}}
\newcommand{\BOSSNet}{\pipeline{BOSSNet}}

\newcommand{\pytorch}{PyTorch}


% math macros
\newcommand{\todo}[1]{\textcolor{orange}{#1}}
\newcommand{\unit}[1]{\mathrm{#1}}
\newcommand{\mps}{\unit{m\,s^{-1}}}
\newcommand{\kmps}{\unit{km\,s^{-1}}}

\newcommand{\teff}{T_\mathrm{eff}}
\newcommand{\logg}{\log_{10}(g)}
\newcommand{\mh}{[\mathrm{M/H}]}


\sloppy\sloppypar\raggedbottom\frenchspacing
\begin{document}

\title{\Huge Astra}

\author[0000-0003-0174-0564]{Andrew R. Casey}
\affiliation{School of Physics \& Astronomy, Monash University}
\affiliation{Centre of Excellence for Astrophysics in Three Dimensions (ASTRO-3D)}

\author{others}

\begin{abstract}

\end{abstract}

\keywords{Foo --- Bar}

\section*{}\clearpage
\section{Introduction}\label{sec:intro}
- In SDSS-V we are using the APOGEE instrument (infrared high res) and the BOSS instrument to observe N many stars as part of MWM
- In SDSS-V we are observing a wider range of stellar types than what was observed in SDSS-IV. Not just giants.
- The typical pipeline that would be suitable for FGKM stars is not suitable for things like white dwarfs.
- This means we end up needing to have multiple pipelines to analyse stars, at least those of different types.
- But since we will have more than one pipeline, we have taken a more elaborate approach to have multiple pipelines for all kinds of stars.
- The reasons for this are:
    - Stellar surveys tend to show substantial systematics with respect to each other
    - 'Allowing room' for multiple different pipelines is intended to encourage the development of new methods.
- This document describes the approach taken for the analysis of stellar parameters in SDSS-V.

\section{Methods}\label{sec:method}


\subsection{Design}

- What does \Astra need to do, other than just to run a loop over all the data?
- Needs to look for new reduced data products (either via on disk, or through a database).
- With each new data product, it records a database entry of that spectrum, so that every spectrum has a unique referenced spectrum index in the database.
- It needs to link that spectrum to an astronomical source. How this is achieved has varied over time, but has since stabilised with the introduction of SDSS ID.
- If we do not already have ancillary metadata for that object, then it needs to find and ingest that data. The ancillary data includes:
    - astrometry;
    - photometry;
    - targeting information;
    - identifiers to other catalogs, etc.
- If this spectrum is a combined spectrum from multiple visits or exposures, then Astra needs to go and find those visits and ingest those spectra. It needs to link those spectra to this combined spectrum.
- Needs to include relevant information from the data reduction pipeline, and any upstream information (e.g., radial velocities)

Why does it need to do all this? 
Because there will be multiple versions of the reduction for the data, and we want to be able to analyse all of those spectra and keep the results together in a database. Every spectrum in the database needs to be able to find the location of the spectrum on disk.

In the end, the output files that \astra\ prepares for a data release will be database exports of subsets of the data (e.g., for a specific data reduction version, range of observation dates, and telescope). 

\subsection{Self-documenting data files}
foo

\subsection{Pipelines}
Here we include a short description of all pipelines included in \astra. 
In general we reference the original papers of those pipelines for detailed explanations, however, we do list any modifications made to those pipelines.
All pipelines have been modified to some degree. In some cases only basic refactoring took place to make the functions compatible with how other pipelines are executed. Some pipelines perform some kind of functionality which we refactored because \astra\ includes that functionality as a common tool, and by refactoring the pipeline we were able to make easy tests of different choices (e.g., how continuum is modelled). In the most extreme cases, the pipeline has been re-written in it's entirety from scratch, whilst keeping the same approach.

\subsection{FERRE} \label{sec:methods-ferre}

FERRE \citep[or FERRE]{ferre} % \todo with the reverse R
is a FORTRAN tool to compare model spectra with observations.
The model spectra should be some rectilinear (evenly spaced) grid of spectra which you want to compare against models.
In the typical use case in \astra, the model grid is convolved to the expected spectral resolution and wavelength sampling of the data.
For a given observed spectrum, the best-fit model is found by interpolating a multi-dimensional grid of spectra.
For computational reasons, and to avoid the so-called `noding' effects \citep{CITE}, the grid of model spectra is stored in a compressed form.

FERRE includes some options when fitting spectra, including the initial guess, and any dimensions to be frozen. For more details see \citet{CITE}.
The continuum is optionally fit simultaneously with the stellar parameters, or can be performed before executing FERRE.
The best-fitting spectrum is found by an optimization routine (Nelder-Mead?).

%Only minimal changes were made to the FERRE code, but more substantive changes were made to the inputs that went in to FERRE (e.g., how ASPCAP is executed).
%Only minimal changes were made to the FERRE code, but more substantive changes were made to how FERRE was executed (e.g., see Section \ref{sec:aspcap-methods}).
The only changes made to FERRE were on the formats of output files. We added the \todo{input name} to output files to ensure that we could correctly order the results from all files when FERRE was being executed in parallel. 
Previously FERRE would correctly sort all files at the end of execution, but this sorting implementation was unexpectedly inefficient. 
After adding the FERRE input name to each row of the output files and the standard output, we disabled the concluding sorts.
Adding the input name to the standard output also allowed \astra\ to better track the time spent analysing per spectrum, and the joint overheads.

FERRE can be executed as a single task in \astra, but it is more often executed as just one step in the analysis.
Custom reader and writer functions to execute FERRE were added to \astra\ to handle the three ways in which FERRE is usually executed: for a coarse estimate of stellar parameters; a detailed fit of all stellar parameters; or by estimating the abundances. These populate different database tables. 

\subsection{ASPCAP} \label{sec:methods-aspcap}

ASPCAP is the APOGEE Stellar Parameter and Chemical Abundances Pipeline \citep{aspcap} that was developed and used by the APOGEE survey in SDSS-IV.\footnote{Those unfamiliar to SDSS may appreciate knowing that there is an APOGEE instrument \emph{and} an APOGEE survey. If not explicitly specified, when referring to APOGEE we mean the APOGEE instrument.}


The ASPCAP and FERRE codes are often described as being synonymous, but there is a clear delineation between the two. FERRE \citep[and Section \ref{sec:methods-ferre}][]{ferre} performs $\chi^2$ minimisation between an observed spectrum and some model grid. FERRE is not usually used to simultaneously estimate both the stellar parameters and detailed chemical abundances. ASPCAP refers to the procedure to estimate stellar parameters and chemical abundances of a star by calling FERRE multiple times.

In the context of SDSS, only APOGEE spectra are analysed with ASPCAP, but ASPCAP has been used for other surveys \citep{who}. 


ASPCAP is probably the most comprehensive pipeline that is integrated in \Astra, and has gone through the most refactoring. Here we will provide a slightly longer-than-average description of ASPCAP so that we can highlight differences in the ASPCAP version in \Astra.

There are three sequential stages in ASPCAP: the coarse stage, the stellar parameter stage, and the chemical abundances stage. The co 



\subsection{The Payne} \label{sec:methods-the-payne}

\subsection{The Cannon} \label{sec:methods-the-cannon}


\subsection{The Classifier} \label{sec:methods-the-classifier}


\subsection{Snow White} \label{sec:methods-snow-white}

The `Snow White' pipeline exclusively analyses white dwarf spectra, which are usually observed with the BOSS 
spectrograph. % todo: how many white dwarfs observed with BOSS compared to APOGEE or both?
`Snow White' measures the equivalent width of \todo{N} lines, and uses a pre-trained random forest \citep{RF} to classify white dwarfs into their sub-types. For white dwarfs that are classified as primarily DA-type, the stellar parameters are fit by interpolating a grid of \citep{who} model spectra and minimising the $\chi^2$ difference between the model and data.

% todos here
- what model grid is used?
- which lines are used?
- continuum normalisation?
- linear interpolation? in how many dimensions
- PCA compression?
- what are the different classification methods and meanings? (there is some structure to this based on the types and probabilistic classifications)

\subsection{MDwarfType} \label{sec:methods-m-dwarf-type}

The `MDwarfType' pipeline classifies M-dwarfs based on an empirical library of \todo{N} spectra of late K- and M-type stars. The empirical library was constructed by Lepine et al. from observations taken at \todo{where}. The \todo{N} spectra have been continuum-rectified by taking the \todo{mean flux between X and Y}.

Most M-dwarf stars in Milky Way Mapper are observed by BOSS (\todo{percent}). We normalise the BOSS observations in the same way that the empirical library has been normalised. We compute the $\chi^2$ difference between the observed spectrum and all \todo{N} templates, and report the best-fitting $\chi^2$ value and template type. The template type includes the spectral classification, as well as a so-called `sub-type' which is used as a coarse proxy for an M-dwarf metallicity.

The `MDwarfType` pipeline is only executed on main-sequence type stars that have been assigned to a carton that is studying M-dwarf stars. That list includes \todo{here}. \todo{Any additional photometric cut that we put on these?}.

% todo citation literature for all the Lepine group stuff.

% todo: a figure of the empirical spectra?

\subsection{HotPayne} \label{sec:methods-hot-payne}

\subsection{APOGEENet} \label{sec:methods-apogee-net}

\APOGEENet\ is a convolutional neural network for estimating stellar parameters ($\teff$, $\logg$, $\mh$) from APOGEE spectra \citep{apogeenet}. Previous versions of \APOGEENet\ required infrared photometry and astrometry (e.g., parallax to compute absolute magnitudes), and in those versions the results were sensitive to missing data. The current version of \APOGEENet\ is version 3, which does not require auxillary photometry. 

\APOGEENet\ is trained on high quality labels from SDSS-IV DR17 \citep{dr17}. 

It is implemented in \pytorch\ and is most efficiently executed on a graphics processing unit (GPU). \Astra\ can readily orchestrate tasks across CPU or GPU clusters, and the default is for \APOGEENet\ to be executed on shared GPU infrastructure at Utah.

\todo{- network structure}

Only minor technical changes were made to the \APOGEENet\ pipeline in \Astra. A task function was written to execute spectra, and we added a load balancer to better handle CPU/GPU bottlenecks.

\subsection{BOSSNet} \label{sec:methods-boss-net}

\BOSSNet\ is a convolutional neural network for estimating stellar parameters ($\teff$, $\logg$, $\mh$) from BOSS spectra \citep{bossnet}. \BOSSNet\ is the natural extension to optical spectra from the same group that developed \APOGEENet. Like the current version of \APOGEENet, no auxillary photometry or astrometry is required by \BOSSNet.

\todo{
- network structure
- training set for bossnet
- testing set
}

\section{Results}\label{sec:results}

- Refer to Don Schneider stuff

- Comments on individual pipeline results? I guess just flagging st


\section{Discussion} \label{sec:discussion}


\section{Conclusions}

\paragraph{Software}
\texttt{numpy} \citep{numpy} ---
\texttt{matplotlib} \citep{matplotlib}.

\paragraph{Acknowledgements}
It is a pleasure to thank
-- people
for valuable discussions and input.

\begin{thebibliography}{dummy}
\bibitem[Kelson(2003)]{kelson} Kelson, D.~D.\ 2003, \pasp, 115, 688. doi:10.1086/375502
\end{thebibliography}

\end{document}
